\hypertarget{background}{%
\chapter{Background}\label{background}}


\hypertarget{ssi-core-standards}{%
\section{SSI core standards}\label{ssi-core-standards}}

The growth of SSI in recent years has been fuelled by the continued effort from World-wide-web consortium (W3C) and Decentralized Identity Foundation (DIF) to develop standards to make SSI applications interoperable.

\hypertarget{decentralised-identifiers-dids}{%
\section{Decentralised identifiers
(DIDs)}\label{decentralised-identifiers-dids}}

DIDs is the standard hosted by W3C here \url{https://www.w3.org/TR/did-core/}. According to DID Use Cases\footnote{\url{https://www.w3.org/TR/did-use-cases/}}, a DID is ``a new type of identifier that has 4 essential characteristics'':

\begin{enumerate}
\def\labelenumi{\arabic{enumi}.}
\tightlist
\item
  \textbf{``Decentralised:} There should be no central issuing agency.''
\item
  ``\textbf{Persistent}: The identifier should be inherently persistent,
  not requiring the continued operation of an underling organisation.''
\item
  \textbf{``Cryptographically verifiable}: It should be possible to prove
  control of the identifier cryptographically.''
\item
  \textbf{``Resolvable}: It should be possible to discover metadata about
  the identifier.''
\end{enumerate}



% ############################################################################
% ------------------------------- PAGE BREAK -----------------------------------
% ############################################################################
\pagebreak





\subsection{DID methods}

There are many different types of DIDs that fit these characteristics. A
specific type of DID is called a DID method. Each DID method provides a
unique specification on how to perform the different
DID method operations - Create, Resolve, Update, Deactivate.

The simplest DID-method is called did:key. did:key is created by
generating a cryptographic public-private keypair on your device, and
using the public part of the keypair as a base for the DID.

\emph{Example of a DID-key:}

\begin{lstlisting}[language=make]
did:key:z6MkpTHR8VNsBxYAAWHut2Geadd9jSwuBV8xRoAnwWsdvktH
\end{lstlisting}

If you stumble upon a DID in the wild, it should be easy to recognise it
as such, because they all share the same format.

\emph{Example of DID format}:

\begin{lstlisting}[language=make]
did:<method>:<unique method-specific string of characters>
\end{lstlisting}

Here is a few more examples of different DIDs from different
DID-methods:

\begin{lstlisting}[language=make]
did:elem:EiAS3mqC4OLMKOwcz3ItIL7XfWduPT7q3Fa4vHgiCfSG2A
did:github:Caranty
did:web:identity.foundation
did:key:z6MkmjY8GnV5i9YTDtPETC2uUAW6ejw3nk5mXF5yci5ab7th
did:sov:WRfXPg8dantKVubE3HX8pw
\end{lstlisting}

As of time of writing all these DIDs are resolveable and you should be
able to resolve them by copy-pasting them into the Universal Resolver hosted here
\url{https://dev.uniresolver.io/}.



\subsection{DID-document} 

When you resolve a DID you will receive a DID-document in return. This
DID-document gives you metadata about the DID. You will need the
DID-document in case you want to interact with the the DID in any form,
because it contains important things like public-keys, for encrypting
the communication and services-endpoints exposing functionality to the
public.

\emph{Example the DID-document of \lstinline!did:sov:WRfXPg8dantKVubE3HX8pw! :}

\begin{lstlisting}
 {
  "@context": [
    "https://www.w3.org/ns/did/v1",
    "https://w3id.org/security/suites/ed25519-2018/v1",
    "https://w3id.org/security/suites/x25519-2019/v1"
  ],
  "id": "did:sov:WRfXPg8dantKVubE3HX8pw",
  "verificationMethod": [{"type": "Ed25519VerificationKey2018", ...},
  "authentication": ["did:sov:WRfXPg8dantKVubE3HX8pw#key-1"],
  "assertionMethod": ["did:sov:WRfXPg8dantKVubE3HX8pw#key-1"],
  "keyAgreement": ["did:sov:WRfXPg8dantKVubE3HX8pw#key-agreement-1"],
  "service": [{"type": "agent", "serviceEndpoint": "https://agents...."}]
}
\end{lstlisting}

\subsection{DID-agent}

The entity which controls a DID, is referred to as an agent. To control
a DID, to become an agent, you must have access to a DIDs private keys.
This will naturally limit who may be able to control a DID, to a single
person, or a group of persons. More on agents later\ldots{}




\hypertarget{didcomm-messaging-didcomm}{%
\section{DIDComm Messaging
(DIDComm)}\label{didcomm-messaging-didcomm}}

DIDComm is the standard hosted by DIF here \url{https://identity.foundation/didcomm-messaging/spec/}.

Now that we have established what DIDs are, we know that each DID has an
agent controlling it. DIDComm is the protocol for how these agents will
communicate together and how they will form networks of agents.

\emph{``The purpose of DIDComm Messaging is to provide a secure, private
communication methodology built atop the decentralized design of DIDs.''
-}\footnote{\url{https://identity.foundation/didcomm-messaging/spec/\#purpose-and-scope}}



\subsection{DIDComm Message envelope} - Each DIDComm message is sealed inside a standard envelope which looks
like this:

\begin{lstlisting}
{
  "typ": "application/didcomm-plain+json",
  "id": "1234567890",
  "type": "<message-type-uri>",
  "from": "did:example:alice",
  "to": ["did:example:bob"],
  "created_time": 1516269022,
  "expires_time": 1516385931,
  "body": {
    "messagespecificattribute": "and its value"
  }
} 
\end{lstlisting}

\url{https://identity.foundation/didcomm-messaging/spec/\#plaintext-message-structure}




% ############################################################################
% ------------------------------- PAGE BREAK -----------------------------------
% ############################################################################
\pagebreak




\subsection{Transports}

A big benefit of DIDComm is that it is transport agnostic. This means that a DIDComm message could be dispatched and delivered across any transport you may see fit. Examples of transports that work with DIDComm include:

\begin{itemize}
\tightlist
\item
  HTTP
\item
  Snail-nail
\item
  Bluetooth
\item
  NFC
\item
  QR codes
\item
  Unix stdin/stdout
\item
  File sharing
\end{itemize}


\subsection{A living standard}

I could write a lot about DIDComm here, but that would be a bit
pointless because the standard-document is very well written, and it is
ever evolving, so my comments on it would quickly get outdated. This is
true for any of the standards mentioned in this paper.




\hypertarget{verifiable-credentials-vcs}{%
\section{Verifiable Credentials
(VCs)}\label{verifiable-credentials-vcs}}

Verifiable Credentials (VCs) is the standard hosted by W3C here \url{https://www.w3.org/TR/vc-data-model/}.

\paragraph{}

We have established that there are agents in the world. In the ideal
SSI-utopia each agent will have a DID attached to them. The agents will
communicate and form networks of agents using DIDComm. The question left
to answer is: What should agents talk aobut? That is where Verifiable
Credentials (VCs) and Verifiable Presentations (VPs) come in.



\subsection{Verifiable data vs plain old data}

Verifiable credentials includes verifiable data. What separates verifiable data from plain old
data? Verifiable data means that you will be able to verify who created
the data in the first place. You are also able to get a
cryptographic-guarantee about the integrity of the data - that no
middle-men has tampered with the data. This is sometimes a good thing,
and other times it is life-critical.

For instance if you have gotten some data about the weather from a
weather API. The API sources it's data from underlying data sources like
\href{http://vy.no}{vy.no} and storm.no. If you receive plain old data
from the API, you cannot know what underlying data source the data is
coming from, just by looking the data. You cannot know if the weather
API has tampered with the data or not. This is maybe not a big issue for
weather reports (may be in some situations), but what if instead of
weather we were talking about your blood type. What if you present some
data about your blood-type, as a doctor you want to be very sure that
that data is coming from a trusted laboratory, before you inject blood
into your patient.

Other examples where verifiable data is important includes passports,
job applications, birth certificates, bank accounts. Wherever there it
makes a difference who created the data, verifiable credentials is a
good candidate technology to use.


\subsection{Issuer, holder, verifier}

Agents may issue, hold and verify Verifiable credentials. Depending on
the role the agent regarding a specific piece of verifiable data, the
agent will either be an issuer, holder and verifier. The same agent may
be all of the above at the same time, depending on what verifiable data
is being processed.





\hypertarget{ssi-applications}{%
\section{SSI Applications}\label{ssi-applications}}

The applications of the SSI-standards range far and wide, but can
generally be categorised into two main categories - SSI agents and SSI
wallets.

\hypertarget{ssi-agents}{%
\subsection{SSI Agents}\label{ssi-agents}}

Applications that performs actions on behalf of on or more DIDs are
referred to as agents. In the real world we find natural agents. People,
organisations, machines, vehicles, anything that has an effect on the
world, or is affected by the world in some ways is an agent. The SSI
agent serves as a digital metaphor for these natural agents. It enables
natural agents to operate as agents in the digital realm as well.

The metaphor agent in the SSI space, is an evolution from the
traditional client-server metaphor. In the ideal SSI world there are no
clients and servers, only agents. The reason why this is important to
notice, is that it forces us to rethink how we interact on the internet.
The client-server view of the world creates a natural imbalance of
power. In web 1.0 and web 2.0 architectures, the server has a
disproportionate amount of power, and the client is subordinate. In the
In the ideal SSI world there are only agents talking to other agents on
an equal playing field. The SSI architecture has the potential to level
the playing field. Because SSI disrupts the traditional architecture of
the web, it is often referred to as web 3.0.

\hypertarget{ssi-wallets}{%
\subsection{SSI Wallets}\label{ssi-wallets}}

What the SSI agent is to natural agents, the SSI wallet is to natural
wallets. SSI wallets attempts to be the digital metaphor for the wallet
you have in your pocket. Verifiable credentials are the things you want
to put in your digital SSI wallet.

Todo: Trinsic, Evernym, Apple Wallet.
