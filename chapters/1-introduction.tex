\chapter{Introduction}



\section{Problem domain - \acrfull{ssi}}

\acrfull{ssi} is a collection of technologies, developed by well-known standards-organizations such as \acrfull{w3c} and \acrfull{dif}.

\paragraph{}
The standards developed within the realm of SSI, are open-source collaborative documents. The documents, when finalized, become specifications which is intended to be used by developers implementing SSI-applications. 

\paragraph{}
\acrfull{dids} is a core \acrshort{ssi}-technology, and it's \textit{design goals}\cite{DIDDesignGoals} gives us a hint about which problems \acrshort{ssi} is trying to solve:
\begin{description}
    \item["Decentralization:] Eliminate the requirement for centralized authorities or single point failure in identifier management, including the registration of globally unique identifiers, public verification keys, services, and other information."
    \item["Control:] Give entities, both human and non-human, the power to directly control their digital identifiers without the need to rely on external authorities."
    \item ["Privacy:] Enable entities to control the privacy of their information, including minimal, selective, and progressive disclosure of attributes or other data."
\end{description}

\paragraph{}
To understand this work, you should at least have a surface-level understanding of the following three core SSI-specifications - \acrfull{dids}, \acrfull{didcomm} and \acrfull{vc} - and how they relate to one another.

\newpage




\subsection{\acrfull{dids}} 

According to \textit{DID Use Cases}\cite{DIDUseCases}, a \acrshort{did} is "a new type of identifier that has 4 essential characteristics:"
\begin{description}
    \item["1. decentralized:] There should be no central issuing agency;"
    \item["2. persistent:] The identifier should be inherently persistent, not requiring the continued operation of an underling organization;"
    \item["3. cryptographically verifiable:] It should be possible to prove control of the identifier cryptographically"
    \item["4. resolvable:] It should be possible to discover metadata about the identifier.
\end{description}
    
\paragraph{}
There are many different types of \acrshort{dids} that fit these characteristics. A specific type of \acrshort{did} is called a \gls{did-method}\cite{DIDMethod}. Each \gls{did-method} provides a unique specification on how to perform the different \gls{did-method-operations}\cite{DIDMethodOperations} - Create, Resolve, Update, Deactivate. 

\paragraph{}
The simplest \gls{did-method} is called \gls{did-key}\cite{DIDKey}. \gls{did-key} is created by generating a cryptographic public-private keypair on your machine, and using the public part of the keypair as a base for the DID.

\begin{lstlisting}[caption={Example of a \gls{did-key}}]
did:key:z6MkpTHR8VNsBxYAAWHut2Geadd9jSwuBV8xRoAnwWsdvktH  <--- Public key
\end{lstlisting}

If you stumble upon a \acrshort{did} in the wild, it should be easy to recognize it as a \acrshort{did}, because they all share the same signature.

\begin{lstlisting}[caption={DID-signature}]
did:<method>:<unique method-specific string of characters>
\end{lstlisting}

\begin{lstlisting}[caption={Depending on the DID-method, different DIDs may look very different from each other.}]
did:elem:EiAS3mqC4OLMKOwcz3ItIL7XfWduPT7q3Fa4vHgiCfSG2A
did:github:Caranty
did:web:identity.foundation
did:key:z6MkmjY8GnV5i9YTDtPETC2uUAW6ejw3nk5mXF5yci5ab7th
did:sov:WRfXPg8dantKVubE3HX8pw

\end{lstlisting}

As of time of writing all these \acrshort{dids} are verified to be resolveable, and you should be able to resolve them by copy-pasting them into \textit{https://dev.uniresolver.io/\cite{UniversalResolver}}.

\newpage




\subsection{\acrfull{didcomm}}

How \acrshort{did}-agents establish connections and pass messages with each other.


Example of a \acrfull{dcpm} \footnote{https://identity.foundation/didcomm-messaging/spec/\#plaintext-message-structure}:
\begin{lstlisting}[language=json]
{
    "id": "1234567890",
    "type": "<message-type-uri>",
    "from": "did:example:alice",
    "to": ["did:example:bob"],
    "created_time": 1516269022,
    "expires_time": 1516385931,
    "body": {
    	"messagespecificattribute": "and its value"
    }
}
\end{lstlisting}

The encrypted variant of \acrfull{dcpm} is called \acrfull{dcem}\footnote{https://identity.foundation/didcomm-messaging/spec/#didcomm-encrypted-message}.



\newpage

\subsection{\acrfull{vc}} 

How \acrshort{did}-agents issue verifiable claims about the attributes of each other and their relationships - https://www.w3.org/TR/vc-data-model/ 

\newpage



\section{Problem Scope - Agent software} 
\begin{itemize}
    \item Within the SSI space, there is a need to develop a specific type of software, called an agent.
    \item An agent is a piece of software which implements protocols defined by the different SSI-specifications.
    \item If an agent follows the specifications correctly, it should be able to communicate interoperably with other agents.
    \item Currently there are SSI-agent-implementations in Python, C#, Javascript, Rust, Golang, and probably more, and they should in theory be interoperable with each other. 
    \item From now on, the term "agent" strictly used to refer to an SSI-agent, NOT web-browser-agent.
    \item Each type of agent implements a different subset of SSI protocols. This is because different agents are made to solve different problems within the SSI problem domain.
    \item For instance a "Wallet", is a specific kind of agent, used to store and manage personal credentials.
\end{itemize}

\paragraph{}
Note: There are many similarities between SSI-agents and web-browser-agents. All web-browsers are interoperable with each other (most of the time), because they are using the same underlying open-source, standardized web-technologies. SSI-agents and web-browser-agents differ in their problem domain, and are not intended to be interoperable with each other.



\section{Product Owner - \acrfull{din}}

Decentalized Identity Norden, or DIN, is a non-profit organization devoted to promote decentralized identity in a nordic context, and highlight the consequences this may have for individuals and society as a whole. For more info see: https://www.din.foundation/om. This work has been developed within the DIN organization.



\section{Project motivation}

DIN wants the project to implement a proof-of-concept agent. The proof-of-concept agent should solve a specific scenario which is relevant to DIN's agenda. The demonstration of solving a specific scenario should inspire and educate people about real-world use-cases of decentralized identity (aka self-sovereign identity).



\newpage

\section{Project Scenario}

The proof-of-concept agent should be able to demonstrate the following scneario.

>The norwegian driver license issuer, Statens Vegvesen, is considering to start issuing it's driver licenses as verifiable credentials (VCs). Statens Vegvesen is not sure if verifiable credentials is the future yet, but are willing to try and dip it's toes in the water. Statens vegvesen will still issue credentials in the traditional way for the forseeable future.

>What Statens Veivesen want is a proof-of-concept SSI-application which will issue, hold and verify driver-licenses. They are hoping that this will bootstrap SSI in Norway, as this will enable other individuals and organizations to start experimenting with holding and verifying a serious credentials which actually are useful.

>The proof-of-concept may demonstrate, that a driver license as a verifiable credential could be considered on par, legally speaking, with traditional driver licenses.

>Statens Vegvesen wants the application to follow open standards which will enable the application to be agnostic about where and how credentials are issued, stored and verified. In other words Statens Vegvesen want to avoid the application to be locked to a specific ledger and a specific wallet.




\section{Goals}

\subsection{Result Goals}

\begin{itemize}
\item Deliver a high-level design document.
\item Develop the application as an open-source project inside DIN-Foundation's Github organization, which solves the scenario.
\item Present and demonstrate that the application implemented solves the scenario.
\item Develop a discussion about interoperability in the SSI ecosystem.
\end{itemize}

\subsection{Effect Goals}

\begin{itemize}
\item Engage with the broader DIF-community during development.
\item Educate people about a practical application of SSI technology.
\end{itemize}

\subsection{Learning Goals}

\begin{itemize}
\item Gain deep understand SSI's layered tech stack - Layer 1 (DID), Layer2 (DIDComm), Layer 3 (VC), Layer 4 (Applications)
\item Gain hands-on experience with developing a SSI agent.
\item Learn a new programming language like Rust.
\item Learn how to work on a project 100% remotely from start to finish.
\end{itemize}





\section{Target audience}

\subsection{Government institutions}

\begin{itemize}
\item The proof-of-concepts is developed specifically to solve the problem of a government institution.
\item The demonstration could inspire lawmakers and government officials to see the true potential of how SSI could make a real change for the better.
\end{itemize}

\subsection{Developers who are new into SSI}
\begin{itemize}
\item When getting into SSI for the first time, the landscape can be a bit difficult to navigate.
\item This project navigates the SSI landscape from a beginners perspective, and by doing so hopes to make the "Getting started"-part for other developers a bit easier.
\end{itemize}

\subsection{SSI Specification writers}
\begin{itemize}
\item Specification writers - People who write the documents that dictate implementations of SSI - are concerned with how beginner-friendly their specifications are.
\item This work may shed some light on how easy it is for new-comers to get started, and may as a result lead to suggested improvements to current specs.
\end{itemize}





\section{Team Roles}

The team working on this project consists of a 1 student, 1 product owner, 2 tech supervisors and 1 academic supervisor.

\begin{table}
  \centering
  \caption{Team Roles}
  \label{tab:example1}
  \begin{tabular}{cc}
    \hline
    Name  & Role \\
    \hline
    Jonas       & Leader, Developer, Student         \\
    Snorre      & Product Owner, Tech supervisor \\
    Mariusz     & Tech supervisor \\
    Abylay      & Tech supervisor \\
    Deepti      & Academic supervisor \\
    \hline
  \end{tabular}
\end{table}



\newpage

\section{Chapters Overview}

\begin{description}
    \item[Chapter 2:] Development Process - How development of DID CLI was managed.
    \item[Chapter 3:] Requirements - The functional and non-functional requirements of DID CLI.
    \item[Chapter 4:] Application Architecture - A description of the high-level design of DID CLI.
    \item[Chapter 5:] Command-Line Interface - A description of the user-interface of DID CLI.
    \item[Chapter 6:] Implementation - Details about DID CLI source-code.
    \item[Chapter 7:] Quality Assurance - How we test and make sure DID CLI does what it is supposed to do. 
    \item[Chapter 8:] Discussion - What did we decide, learn, think, feel?
    \item[Chapter 9:] Conclusion - Summary and future work.
\end{description}
