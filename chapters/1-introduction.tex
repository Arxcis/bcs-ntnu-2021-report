\hypertarget{introduction}{%
\chapter{Introduction}\label{introduction}}

\hypertarget{ssi-background}{%
\section{Self-Sovereign Identity (SSI)}\label{ssi-background}}

On April 25th 2016, Chirstopher Allen launched a blog-post laying out
his vision for the future of digital identity titled
\href{http://www.lifewithalacrity.com/2016/04/the-path-to-self-soverereign-identity.html}{``The
Path to Self-Sovereign Identity(SSI)''}. Allen's vision sees SSI as the next
step in the evolution of digital identity: 

``The models for online
identity have advanced through 4 broad stages since the advent of the
Internet:

\begin{itemize}
\tightlist
\item
  Stage 1: Centralized identity
\item
  Stage 2: Federated identity
\item
  Stage 3: User-centric identity
\item
  Stage 4: Self-sovereign identity'' - Allen
\end{itemize}

Decentralised Identifiers (DIDs) is a core SSI-technology, and it's
design goals gives us a hint about which problems SSI as a whole is
trying to solve:

\begin{itemize}
\tightlist
\item
  ``\textbf{Decentralization}: Eliminate the requirement for centralized
  authorities or single point failure in identifier management,
  including the registration of globally unique identifiers, public
  verification keys, services, and other information.''
\item
  ``\textbf{Control}: Give entities, both human and non-human, the power
  to directly control their digital identifiers without the need to rely
  on external authorities.''
\item
  ``\textbf{Privacy}: Enable entities to control the privacy of their
  information, including minimal, selective, and progressive disclosure
  of attributes or other data.''
\end{itemize}


\paragraph{}

\textbf{SSI and Web 3.0} - Also it's worth recognizing SSI as part of a greater ecosystem, often referred to as Web 3.0. Web 3.0 includes blockchains (Bitcoin, Ethereum, etc.), decentralised databases (IPFS, Arbitrum, etc.), Decentralised finance (DeFi), and more.




% ############################################################################
% ------------------------------- PAGE BREAK -----------------------------------
% ############################################################################
\pagebreak




\hypertarget{ssi-and-web-3.0-in-the-news}{%
\section{SSI and Web 3.0 in the
news}\label{ssi-and-web-3.0-in-the-news}}

Although SSI is not part of the general publics awareness at this point,
some forward-leaning public institutions have started discussions about
how to draw benefits from SSI technologies and Web 3.0 as a whole:

\begin{itemize}
\tightlist
\item
    \textbf{EU, eIDAS and SSI} - ``SSI and eIDAS: a vision on how they are connected - Share your views!'' \footnote{\url{https://ec.europa.eu/futurium/en/eidas-observatory/ssi-and-eidas-vision-how-they-are-connected-share-your-views.html}}
\item
    \textbf{Canada and SSI} - ``The Future of Digital Identity in Canada'' \footnote{\url{https://www.frontiersin.org/articles/10.3389/fbloc.2021.624258/full}}
\item
    \textbf{Bank of America} - ``Bank of America Sees DeFi `Potentially More Disruptive Than Bitcoin'' \footnote{\url{https://finance.yahoo.com/news/bank-america-sees-defi-potentially-164335888.html?guccounter=1}}
\end{itemize}

\paragraph{}

``It's difficult to overstate how transformative blockchain
technology, digital assets and the thousands of decentralized apps that
have yet to be created could potentially be.'' - Bank Of America\footnote{\url{https://coinmarketcap.com/alexandria/article/crypto-too-large-to-ignore-bank-of-america-says}}




\hypertarget{project-team}{%
\section{Project Team}\label{project-team}}

\hypertarget{team-roles}{%
\subsection{Team Roles}\label{team-roles}}

The team working on this project consists of a 1 student, 1 product
owner, 2 tech supervisors and 1 academic supervisor.

\begin{itemize}
\tightlist
\item
    Jonas - Project lead, Student, Rust developer
\item
    Snorre - Project initiator, Client, Product owner
\item
    Deepti - Academic supervisor
\item
    Mariusz - Technical supervisor
\item
    Abylay - Technical supervisor
\end{itemize}


\hypertarget{project-initiator}{%
\subsection{Project Initiator}\label{project-initiator}}

Snorre Lothar von Gohren is the project's initiator. He is the CTO \& Co-Founder of Diwala, which is a for-profit company set out to ``build an ecosystem of digital skill identities with verifiable credentials''.

Snorre also founded DIN - a network of nordic organisations which ``works to
promote decentralised identity and highlight how individuals and society
is affected by it.



% ############################################################################
% ------------------------------- PAGE BREAK -----------------------------------
% ############################################################################
\pagebreak



\hypertarget{project-description}{%
\section{Project description}\label{project-description}}

\emph{``The project ambition is to build a proof with the existing
wallet standards, that it is possible to establish an open and
interoperable identity wallet. Or even prove that this is not possible
at this moment, because of the lack of certain standards.}

\emph{As part of the bachelor project team will develop a set of
proof-of-concept reference implementations that can showcase the
interoperability within the existing frameworks and tools that are
already out there. Set together the needed pieces to prove that
interoperability is possible. The reference implementation of the
proof-of-concept, will prove that the user will be able to seamlessly
respond to different service proof requests and issuance requests, and
to be able to migrate credentials and other user-centric data stored in
the users control.}

\emph{The work will begin with existing open source frameworks, as well
as closed source service providers with good developer portals for easy
testing and verification. The project is language and platform agnostic.
The biggest benefit in this is to showcase that the proclaimed panacea
for digital identity, is actually possible. This work will rapport on
hinders, benefits and possibilities with this ecosystem, and the
importance of interoperability''}

\hypertarget{project-scope}{%
\section{Project scope}\label{project-scope}}

\hypertarget{target-audience}{%
\subsection{Target audience}\label{target-audience}}

The people we want to target are public institutions in Norway. Public
institutions like the Police, Statens Vegvesen, Brønnøysundregisterene,
Folkeregisteret, Skatteetaten and others are all in a position to
utilise SSI tech, because they are guardians of many key pieces of
personal data linked to the citizens of Norway. If these institutions
were to start issuing verifiable data as Verifiable Credentials, they
might be able to bootstrap SSI in Norway and rocket the Norwegian
digital economy into the Web 3.0 and SSI.


\hypertarget{goals}{%
\subsection{Goals}\label{goals}}

\begin{itemize}
\tightlist
\item
  Create a proof-of-concept SSI application to demonstrate the
  implementation of a scenario involving real public institutions in
  Norway.
\item
  Inspire public institutions to adopt SSI tech.
\item
  Use bleeding-edge SSI standards.
\item
  Keep scope as focused and as tiny as possible, because there is only 1
  student on this project.
\end{itemize}




% ############################################################################
% ------------------------------- PAGE BREAK -----------------------------------
% ############################################################################
\pagebreak



\hypertarget{scenario}{%
\subsection{Scenario}\label{scenario}}

To motivate the development and limit scope we want to define a specific
scenario which our proof-of-concept should be the solution to. The
real-world scenario depicted below is just for demonstration purposes
and not an actual scenario:

\paragraph{}

\emph{The Norwegian drivers license issuer, Statens Vegvesen, is
considering to start issuing its drivers licenses as VCs. Statens
Vegvesen is not sure if VCs are the future yet, but are willing to try
and dip its toes into the water.}

\emph{What Statens Veivesen want is a proof-of-concept application to be
developed, which will issue, hold and verify driver-licenses as VCs. The
proof-of-concept should demonstrate how it handles a situation were a
driver is pulled over by the police.}

\paragraph{}

\emph{\textbf{Agents involved}:}

\begin{itemize}
\tightlist
\item
  \emph{VC issuer - Statens Vegvesen}
\item
  \emph{VC verifier - Police}
\item
  \emph{VC holder - Driver}
\end{itemize}

\hypertarget{organisational-boundaries}{%
\subsection{Organisational boundaries}\label{organisational-boundaries}}

\begin{itemize}
\tightlist
\item
  Source-code should be open-source.
\item
  Source-code should be hosted on \emph{github.com/DIN-Foundation.}
\end{itemize}

\hypertarget{technical-boundaries}{%
\subsection{Technical boundaries}\label{technical-boundaries}}

\begin{itemize}
\tightlist
\item
  Application should be implemented in the Rust programming language.
\item
  Use existing Rust libraries wherever possible.
\item
  The user interface should be a CLI - no GUI.
\item
  Only support one DIDComm transport: STDIN/STDOUT
\item
  Only support one DID-method: did-key
\item
  Only support one cryptographic-method: x25519/ed25519
\end{itemize}



% ############################################################################
% ------------------------------- PAGE BREAK -----------------------------------
% ############################################################################
\pagebreak



\hypertarget{chapters}{%
\section{Chapters}\label{chapters}}

\begin{enumerate}
\def\labelenumi{\arabic{enumi}.}
\tightlist
\item
  \textbf{Introduction} -Project description, scope, team, the DID CLI.
\item
  \textbf{Background} - Background documentation explaining core technologies.
\item
  \textbf{Development Process -} How we developed DID CLI and this
  report.
\item
  \textbf{Functional Requirements -} Listing functional requirements of DID-CLI.
\item
  \textbf{User Interface -} Documenting the user interface (UI/UX) of DID-CLI.
\item
  \textbf{Architecture -} An overview of DID-CLI architecture and details about each component.
\item
  \textbf{Results -} Demo of using DID-CLI to solve the scenario.
\item
  \textbf{Discussion -} What did we decide, learn, think, feel?
\item
  \textbf{Conclusion -} Summary and future work.
\end{enumerate}
