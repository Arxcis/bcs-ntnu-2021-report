\chapter{Command-Line Interface}

\begin{itemize}
\item The main way to interact with `did` executeable, is through it's CLI.
\item Each command follows the same pattern `did <command> <...args>`.
\item The DID-CLI draws inspiration from the book `The Unix Programming environment` by `Brian W. Kernighan` and `Rob Pike`, 1984.
\end{itemize}

\section{Unix Pipelines Integration}

Many of the commands in the DID-CLI, is designed in a way to easily integrate with existing Unix tools. The most important part of this integration, is to support optionally reading input from `stdin`. Also it is important to take care in how output is written to `stdout`, to make it possible to chain commands together. This is the reason you will see that most of commands are standardized to write full DCEM-messages to `stdout`, for easy consumption by the next command in the pipeline.

*Example:*
```
cat message.dcem | did read | grep jonas 
```


\section{Commands}

\subsection{did help}

- List all available commands.

- **Example:**

    ![](./images/cmd-help.png)

\subsection{did init}

- Initializes a did-agent in the working directory.
- Run this command before running any other commands.
- The command creates a new `.did/`-directory, inside your working directory.
- A secret/private key is stored inside `.did/`.
- All your agents wallet-data will be stored inside `.did/`.
- Your agents `did` will be returned to `stdout` when running this command.
- If a `.did/` already exists, this commands has no side-effects - the command is idempotent.
- `did init` is intentionally almost identical to `git init`, to make it easier for new users to understand the CLI by reusing a good design-pattern from a well-known CLI like `git`-CLI.

- **Example**:

    ![](./images/cmd-init.png)


\subsection{did doc}

- Prints the did-document, controlled by the did agent.
- Since the did-agent uses did-key as it's underlying did-method, the did-document is generated from the public-private keypair.
- Another way to describe this is that did-key is self-resolving - the did-document is resolved directly from the did.
- This is a limitation of the did-key method, and how it is specified.
- Once created, the did-document pinned to a did-key did, is not possible to edit.

- **Example**:

    ![](./images/cmd-doc.png)


\subsection{did dids}

- List all dids stored in the agent.
- Dids are added to the agent when running the `did connect` command.

- **Example:**

    ![](./images/cmd-dids.png)

\subsection{did did <didname>}

- Show the did of a single `<didname>`.

- **Example:**

    ![](./images/cmd-did.png)


\subsection{did connect <didname> <did>}

- `did connect` connects a `<didname>` to `<did>`
- `did connect` gives a `<did>` a `<didname>`.
- The `<didname>` is used in other commands, as an easy way to refer to another agent's `<did>`.

- **Example**:

    ![](./images/cmd-connect.png)

\subsection{did write <didname> <message>}

- Wraps a user defined message inside a `<dcem>`-envelope.
- Sets the `to`-header of the `<dcem>` to the underlying `<did>` refered to by the `<didname>`.
- Gives the message a new globally unique `id`.

- **Example**:

    ![](./images/cmd-write.png)

    ![](./images/cmd-write-alt.png)

\subsection{did read <dcem>}

- Unwraps an `<dcem>` message from `stdin` or from `<dcem>`-arg.
- Prints the plaintext body of the message.

- **Example**:

    ![](./images/cmd-read-message.png)

    ![](./images/cmd-read-vc.png)

    ![](./images/cmd-read-vp.png)


\subsection{did issue <CredentialType> <didname>}

- Issues a verifiable credential addressed to the `did` of `<didname>`:
- Issues one of 4 `<CredentialType>`s
    * Passport
    * DriversLicense
    * TrafficAuthority
    * LawEnforcer

- **Example**:

    ![](./images/cmd-issue.png)

    ![](./images/cmd-issue-alt.png)


\subsection{did hold <dcem>}

- **Example:**

    ![](./images/cmd-hold.png)

\subsection{did present <didname> <dcem>}

- **Example:**

    ![](./images/cmd-present.png)

    ![](./images/cmd-present-alt.png)

\subsection{did verify <issuer didname> <subject didname> <dcem>}

- Print `<dcem>` to `stdout`, if, and only if, verification succeeds.

- **Example:**

    ![](./images/cmd-verify.png)

    ![](./images/cmd-verify-issuerfails.png)

    ![](./images/cmd-verify-subjectfails.png)

\subsection{did messages}

- List all didcomm messages stored in the wallet.
- Messages are added to the wallet when using the `did hold` command.

- **Example:**

    ![](./images/cmd-messages.png)

\subsection{did message <message id>}

- Show the contents of a single didcomm message based on the given `<message id>`.

- **Example:**

    ![](./images/cmd-message.png)


\section{Intentional limitations of the CLI}

\begin{itemize}
\item None of the commands have any optional-arguments - e.g `--option=<arg>`. This is to keep program logic as simple as possible. If the CLI was intended for a broader audicene with multiple use-cases, options may be added. This CLI is a special purpose CLI, intended to solve a specific use-case, namely the specific proof-of-concept from the problem statement. This is why optional-arguments was not prioritized.
\item Options are much harder to parse correctly than fixed size positional arguments.
\item None of the commands required variable length arguments, which made the implementation easier.
\item None of the commands have filepath arguments. The user is expected to use `cat <filepath>` to read the contents of a file, which is then fed into a positional argument of one of the commands. Example: `did read \$(cat ../message.dcem)` vs `did read ../message.dcem`. This was done to simplify implementation.
\end{itemize}