\hypertarget{discussion}{%
\chapter{Discussion}\label{discussion}}

This chapter will be written with "I" instead of "WE", since there is only a single author, and this chapter will get a bit more subjective than the previous chapters.


\section{The Challenge of Academic writing}

By far the biggest challenge I faced in this project has been: "How to write a bachelor?". When selecting between different projects, I was looking for a project which could bring me a decent challenge. What I did not know at that point in time was that no matter which project I would have chosen, the difficulty of the problem description would have been dwarfed by how difficult it has been for me to write an academic paper. Why is it so difficult for me?

The first an most obvious point to make is that writing an academic paper is difficult for me because I have never ever done it before. I am not used to think about a problem in the way which tends well to academic writing. I had supervisors along the way which were supposed to guide me in the correct direction, but it is very difficult to get supervision when I don't even have the concepts and vocabulary about academic writing inside my head.

Slowly but surely I have navigated down the list of chapters, forming new pathways in my brain of what goes where. At this point in time I think I have gotten a fairly good grasp on what the concepts of Abstract, Acknowledgements, Introduction, Background, Methodology (development process, functional requirements, user interface and architecture in this paper) and Results mean, meaning I know what to put where. This is not a given. It took a long time, many many months, before I fully understood the difference between Introduction, Background and Methodology for instance. It did not help very much that the supervisors told me again and again what it meant, because I never fully understood it on a deeper lever, because I have never been through the exercise of trying to actually put things into the different buckets.

Writing an academic paper is a highly practical skill. You cannot simply read how to do it and then do it correctly. One has to learn it by trial and error. I have done so many errors. The background-chapter for instance did not exist until a couple of days before the deadline, but when I first understood why I needed the Background chapter and what I should put in it, it was like 100 kilos was lifted off my shoulders. I have had many moments such as this. The Abstract was completely empty for many many months and I did not understand what it was for, until one day - eureka! - I spammed the keyboard for twenty-minutes and all the pieces fell into the right places.

To sum it up, the challenge with this bachelor has been 90 percent "How to write a bachelor?", 5 percent "What the heck is SSI?" and 5 percent "How do I create a CLI in Rust?".

Has all this struggle scared me away from academic writing in the future? No, I really don't think so. I'm actually looking forward to my next chance at writing an academic paper, because I will have learned so much for my mistakes here. I will spend so much less time on fighting with Latex, battling glossary, bibliographies, moving chapters around, writing entire pages of text just to delete them the next day. So much effort will be saved because I actually understand the point of all this. Hopefully I will be able to use all this saved effort into solving the actual problem instead, which is why I am looking forward to do this again.



\section{Did we solve anything from the problem description?}

The main problem put forth initially by Snorre (project initiator), and reiterated throughout the project's development, was to investigate interoperability issues within the SSI ecosystem. I have always been onboard with this viewpoint. I have understood the problem very well. Therefore it is with a heavy heart that I have to announce that we never really got this project to a point where interoperability could be investigated.

Was all the work put down in this project for nothing then? No, as is evident from the results chapter, we did end up with some results, but it is an understatement to tell you that the results we got was a but disappointing. We had hoped to get further than what we did. Of course, it was not all a waste of time, but as discussed in the previous section, most of the time and effort in this project have been spent trying to learn the author of this document on "how to write a paper" and not so much have gone into actually working with SSI related things.



\section{Why are citations, bibliography, glossary and acronyms missing in this paper?}

As mentioned in the previous two sections, it has been difficult for the author of this document to actually understand all the concepts related to writing a document like this. Due to this limitation and limited time the author of this document decided to reduce the amount of complexity, to actually finish the basics.

Juggling around with citations, bibliography, glossary and acronyms is not easy, while at the same time trying to understand basic things like: "What text goes in which chapter?", "What does even Methodology mean?", "Why does every bachelor paper look different?", "How should my paper look like?". The choice of removing the complexity these things brought to the development of this paper, freed up a lot of energy which could be put into actually completing the more basic things.

I am really happy that made this decision, because it made it possible for me to get a lot further than what it seemed like at some point. What is sad, is that this means that this paper will never really be accepted as a proper academic publication, because it does not follow "good form". This is fine, because I will be painfully aware of this next time around.



\section{Was it worth it not delivering this paper on time?}

This bachelor thesis was supposed to be delivered on 20.05.2021. It was actually delivered 01.12.2021 - a whole semester behind schedule. The author of this paper actually needed a full extra semester of completing this task, and the reasons for this are laid out in the previous sections. Let's go back to the question of this section then. Was it worth it, not delivering this paper on time?

The specific reasons for not delivering on time, was that vital components of this paper were not written. The chapters that were present on 20.05.2021 were Introduction, Development Process, Requirements and User Interface. The chapters that were missing on 20.05.2021 were Architecture, Results, Discussion, Conclusion, Abstract, Appendix and Acknowledgment. All these chapters are present in the paper today, 30.11.2021. The things that are still missing are laid out in section 8.3.

This leads me to conclude that it was totally worth it for me to NOT deliver the paper on the original deadline. The improvements I have been able to make have been substantial. The learning I have been able to achieve has been huge. Having said that, it has cost me a lot of pain, not having completed this for a full semester. Spending 1 year on a bachelor, is 6 months more pain that what was really necessary, but in the end it was worth it, given the circumstances.




\section{Little to no peer-review}

The only section of this paper which have gone through some level of scrutiny by a third party, before being delivered, is the Abstract chapter. It is not coincidence that this chapter is the chapter I am most satisfied with. The reason I haven't received any peer review on any of the other chapters are not due to a lack of reviewers though. I have had loads of potential reviewers around me supporting me throughout the project. The problem is that during most of the project I have not been ready to accept feedback, because the paper had not gotten to a state where I felt comfortable with scrutinizing it.

Nearing the end of developing this paper I finally feel that I am ready to welcome people from the outside into the paper to give constructive feedback. The problem is that I reached this point less than 48 hours before the deadline. This is of course way too late to reach the "review"-territory. Many of the reasons this has happened are laid out in the previous sections of this chapter.

It really is a pity, because I really do love peer-review of whatever I am working on, be it code, documents or music. It really motivates me to take things to the next level of quality and clarity. This again becomes one more reason to look forward to future projects, where I will be able to bring in reviewers a lot earlier in the process.