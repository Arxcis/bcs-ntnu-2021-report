\hypertarget{discussion}{%
\chapter{Discussion}\label{discussion}}

This chapter will be written with "I" instead of "WE", since there is only a single author, and this chapter will get a bit more subjective than the previous chapters.


\section{The Challenge of Academic writing}

By far the biggest challenge I faced in this project has been: "How to write a bachelor?". When selecting between different projects, I was looking for a project which could bring me a decent challenge. What I did not know at that point in time was that no matter which project I would have chosen, the difficulty of the problem description would have been dwarfed by how difficult it has been for me to write an academic paper. Why is it so difficult for me?

The first an most obvious point to make is that writing an academic paper is difficult for me because I have never ever done it before. I am not used to think about a problem in the way which tends well to academic writing. I had supervisors along the way which were supposed to guide me in the correct direction, but it is very difficult to get supervision when I don't even have the concepts and vocabulary about academic writing inside my head.

Slowly but surely I have navigated down the list of chapters, forming new pathways in my brain of what goes where. At this point in time I think I have gotten a fairly good grasp on what the concepts of Abstract, Acknowledgements, Introduction, Background, Methodology (development process, functional requirements, user interface and architecture in this paper) and Results mean, meaning I know what to put where. This is not a given. It took a long time, many many months, before I fully understood the difference between Introduction, Background and Methodology for instance. It did not help very much that the supervisors told me again and again what it meant, because I never fully understood it on a deeper lever, because I have never been through the exercise of trying to actually put things into the different buckets.

Writing an academic paper is a highly practical skill. You cannot simply read how to do it and then do it correctly. One has to learn it by trial and error. I have done so many errors. The background-chapter for instance did not exist until a couple of days before the deadline, but when I first understood why I needed the Background chapter and what I should put in it, it was like 100 kilos was lifted off my shoulders. I have had many moments such as this. The Abstract was completely empty for many many months and I did not understand what it was for, until one day - eureka! - I spammed the keyboard for twenty-minutes and all the pieces fell into the right places.

To sum it up, the challenge with this bachelor has been 90 percent "How to write a bachelor?", 5 percent "What the heck is SSI?" and 5 percent "How do I create a CLI in Rust?".

Has all this struggle scared me away from academic writing in the future? No, I really don't think so. I'm actually looking forward to my next chance at writing an academic paper, because I will have learned so much for my mistakes here. I will spend so much less time on fighting with Latex, battling glossary, bibliographies, moving chapters around, writing entire pages of text just to delete them the next day. So much effort will be saved because I actually understand the point of all this. Hopefully I will be able to use all this saved effort into solving the actual problem instead, which is why I am looking forward to do this again.
