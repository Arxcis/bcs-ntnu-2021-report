\chapter*{Abstract}

This project concerns the development of Decentralized Identity Command-Line Interface(DID CLI), which is a proof-of-concept application implementing Self-Sovereign Identity(SSI) written in the Rust programming language.

The Self-Sovereign Identity technology stack(SSI stack) describes 4 layers of technologies required to implement SSI applications. Layer 1(DIDs) concerns decentralised identifiers. Layer 2(DIDComm) concerns communication between decentralised agents. Layer 3(Verifiable Credentials) concerns the exchange of cryptographically verifiable data. Layer 4(Governance) concerns using the technologies from layer 1, 2 and 3, to solve real world problems. In addition, we propose layer 0 (Cryptography) concerning the cryptographic primitives underpinning it all. 


DID CLI demonstrates that open-source libraries already exist in Rust for all the SSI layers mentioned above. Layer 0(Cryptography), limited in this project to the ED25519 elliptic-curve signature, is implemented using \newline \textit{github.com/dalek-cryptography/ed25519-dalek}. Layer 1(DIDs), limited in this project to the \textit{did:key-method}, is implemented using \textit{github.com/decentralized-identity/did-key.rs}. Layer 2(DIDComm), is implemented using \newline \textit{github.com/decentralized-identity/didcomm-rs}. Layer 3(Verifiable credentials), is implemented using \textit{github.com/spruceid/ssi}. Finally DID CLI itself is the manifestation of layer 4(Governance).


DIDComm v2 is the next iteration of a layer 2 technology, developed by Decentralised Identity Foundation (DIN). Being in early stages of development, there are very few open-source examples of applications which implement DIDComm v2, that we know of. Based on the experience we gained, developing DID CLI, we were able to provide practical feedback directly to the DIDComm v2 specification writers, which they embraced with open arms.


HTTPS, Websockets and Bluetooth are all mentioned as viable transports in the DIDComm v2 specification, but in general DIDComm v2 claims to be transport agnostic. With DID CLI we take this claim to it's extreme conclusion and implement DIDComm over Unix Standard In/Standard Out (STDIN/STDOUT). DID CLI demonstrates that DIDComm over STDIN/STDOUT is not only viable, but desireable, when integrating with the long-lived Unix ecosystem. We demonstrate that by using DID CLI, 4 people are able to form a network of SSI agents, communicating with each other via a Unix filesystem, solving a real world scenario.