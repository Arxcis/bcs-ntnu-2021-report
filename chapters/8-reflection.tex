\chapter{Reflection}

\section{Joining a large existing open-source community as a complete noob}

I spent some time in the beginning to get in touch with community experts. This resulted in me joining the DIF foundation, to get access to it's Slack-channel, and ask direct questions to the people that know the most about SSI, and it's development, in the world.

\section{Learning a new low-level language - Rust}

I had a lot of experience from programming in low-level languages and was surprised on how different it felt to program in Rust compared to C, C++, which are very similar to each other. Rust shows that inovation is still possible in this area, after many years with most innovation happening in higher-level languages.

\section{Techniques to overcome writers block}

- Write something every day, no exceptions. Even if it just a single word. Do not stop ever.
- Allow yourself to write stupid things, and then rewrite. Rewriting is easier than writing.
- Write bullet points.
- Embrace chaos.
- Don't be afraid to delete and move things around.

\section{How to talk to regular people about Self-Sovereign Identity?}

A list of concepts regular-people understand:
- Attester
- Vitnemål
- Kursbevis
- Kontrakter
- ID
- Skjøter
- Titler
- Fødselsattest
- Pass
- BankID
- Førerkort
- Bankkontoer
- Lånebevis
- Penger (cryptocurrency)
- Helsejournal - 200 different countries, 200 different helth care apps?
- Resepter
- ...and more
