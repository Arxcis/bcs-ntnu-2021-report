
% From https://www.overleaf.com/learn/latex/Glossaries

\makeglossaries % Prepare for adding glossary entries


\newglossaryentry{did-method}
{
        name=DID-method,
        plural=DID-methods,
        description={A specific type of DID. Each DID-method has it's own implementation of the DID-method operations}
}
\newglossaryentry{did-method-operations}
{
        name=DID-method Operations,
        description={Create, resolve, update, deactivate}
}
\newglossaryentry{did-key}
{
        name=did:key,
        description={The simplest DID-method. It is based on a cryptographic public-private keypair}
}
\newglossaryentry{unix}
{
        name=UNIX,
        description={A programming environment for terminals, with initial release in 1971}
}

% --------------------
% ----- Acronyms -----
% --------------------

\newacronym{ssi}{SSI}{Self-Sovereign Identity}
\newacronym{did}{DID}{Decentralized Identifier}
\newacronym{dids}{DIDs}{Decentralized Identifiers}
\newacronym{w3c}{W3C}{World Wide Web Consortium}
\newacronym{dif}{DIF}{Decentralized Identity Foundation}
\newacronym{din}{DIN}{Decentralized Identity Norden}

\newacronym{didcomm}{DIDComm}{DIDComm Messaging v2}
\newacronym{vc}{VC}{Verifiable Credentials}
\newacronym{vp}{VP}{Verifiable Presentations}
\newacronym{bdd}{BDD}{Behaviour Driven Development}

\newacronym{dcpm}{DCPM}{DIDComm Plaintext Message}
\newacronym{dcem}{DCEM}{DIDComm Encrypted Message}

\newacronym{cli}{CLI}{Command Line Interface}
\newacronym{did-cli}{DID-CLI}{Decentralized Identity Command Line Interface}

\newacronym{stdin}{STDIN}{Unix Standard Input}
\newacronym{stdout}{STDOUT}{Unix Standard Output}